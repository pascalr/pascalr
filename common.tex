\usepackage[utf8]{inputenc}
\usepackage[T1]{fontenc}
\usepackage{hyperref}
\usepackage{todonotes}
\usepackage{gensymb} % superscript
\usepackage{textcomp} % superscript
\usepackage{amsmath} % symboles des degrés

% --------------------  NOMBRES ET UNITÉS ---------------------------
\usepackage{siunitx}
\iffalse % explication des fonctions courantes
The package provides the user macros:
• \ang[hoptionsi]{hanglei}
• \num[hoptionsi]{hnumberi}
• \si[hoptionsi]{huniti}
• \SI[hoptionsi]{hnumberi}[hpre-uniti]{huniti}
• \numlist[hoptionsi]{hnumbersi}
• \numrange[hoptionsi]{hnumbersi}{hnumber2i}
• \SIlist[hoptionsi]{hnumbersi}{huniti}
• \SIrange[hoptionsi]{hnumber1i}{hnumber2i}{huniti}
• \sisetup{hoptionsi}
• \tablenum[hoptionsi]{hnumberi}
plus the S and s column types for decimal alignments and units in tabular environments.
12 345.678 90        \num{12345,67890}
1 ± 2i               \num{1+-2i}
0.3 × 1045           \num{.3e45}
1.654 × 2.34 × 3.430 \num{1.654 x 2.34 x 3.430}

\si{kg.m.s^{-1}}
kg m s−1
Simple lists and ranges of numbers can be handled.
\numlist{10;20;30} \\
\SIlist{0.13;0.67;0.80}{\milli\metre} \\
\numrange{10}{20} \\
\SIrange{0.13}{0.67}{\milli\metre}
10, 20 and 30
0.13 mm, 0.67 mm and 0.80 mm
10 to 20
0.13 mm to 0.67 mm

\ohm\metre
\celsius

\fi
% --------------------------------------------------------------

\usepackage{hyperref}

\usepackage{natbib}       % Pour la bibliographie
\usepackage[french]{babel}
\usepackage{graphicx}     % Pour les figures
\usepackage[toc]{appendix}% Pour les annexes
\usepackage{titlesec}     % Pour modifier les titres
\usepackage{fancyhdr}     % Pour les haut et bas de pages (numéros)
\usepackage{lipsum}       % Rajoute du texte temporaire
\usepackage{parskip}      % Modifie l'apparence des paragraphes
\usepackage{setspace}     % Pour interligne et demi
\usepackage{listings}     % Pour afficher du code sans formattage

\selectlanguage{french}

% ------------------------- COMMANDES -------------------------------

\newcommand\var[2]{\newcommand{#1}{\ensuremath{#2}}}
% C'est pratique de définir des variables au début pour éviter de
% réécrire des formules et cela facilite modifier l'affichage.

\makeatletter
\newcommand{\makecustomtitle}{
  \newpage
  \null
  \begin{center}
  \let \footnote \thanks
    {\LARGE \textbf{\@title} \par}
    \vskip 1em
    {\large \@date \par}
    \rule{1.5in}{0.4pt}
  \end{center}
  \par
}
\makeatother
