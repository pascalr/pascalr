\documentclass{article}
\usepackage[utf8]{inputenc}
\usepackage[T1]{fontenc}
\usepackage{hyperref}
\usepackage{todonotes}
\usepackage{gensymb} % superscript
\usepackage{textcomp} % superscript
\usepackage{amsmath} % symboles des degrés

% --------------------  NOMBRES ET UNITÉS ---------------------------
\usepackage{siunitx}
\iffalse % explication des fonctions courantes
The package provides the user macros:
• \ang[hoptionsi]{hanglei}
• \num[hoptionsi]{hnumberi}
• \si[hoptionsi]{huniti}
• \SI[hoptionsi]{hnumberi}[hpre-uniti]{huniti}
• \numlist[hoptionsi]{hnumbersi}
• \numrange[hoptionsi]{hnumbersi}{hnumber2i}
• \SIlist[hoptionsi]{hnumbersi}{huniti}
• \SIrange[hoptionsi]{hnumber1i}{hnumber2i}{huniti}
• \sisetup{hoptionsi}
• \tablenum[hoptionsi]{hnumberi}
plus the S and s column types for decimal alignments and units in tabular environments.
12 345.678 90        \num{12345,67890}
1 ± 2i               \num{1+-2i}
0.3 × 1045           \num{.3e45}
1.654 × 2.34 × 3.430 \num{1.654 x 2.34 x 3.430}

\si{kg.m.s^{-1}}
kg m s−1
Simple lists and ranges of numbers can be handled.
\numlist{10;20;30} \\
\SIlist{0.13;0.67;0.80}{\milli\metre} \\
\numrange{10}{20} \\
\SIrange{0.13}{0.67}{\milli\metre}
10, 20 and 30
0.13 mm, 0.67 mm and 0.80 mm
10 to 20
0.13 mm to 0.67 mm

\ohm\metre
\celsius

\fi
% --------------------------------------------------------------

\usepackage{hyperref}

\usepackage{natbib}       % Pour la bibliographie
\usepackage[french]{babel}
\usepackage{graphicx}     % Pour les figures
\usepackage[toc]{appendix}% Pour les annexes
\usepackage{titlesec}     % Pour modifier les titres
\usepackage{fancyhdr}     % Pour les haut et bas de pages (numéros)
\usepackage{lipsum}       % Rajoute du texte temporaire
\usepackage{parskip}      % Modifie l'apparence des paragraphes
\usepackage{setspace}     % Pour interligne et demi
\usepackage{listings}     % Pour afficher du code sans formattage

\selectlanguage{french}

% ------------------------- COMMANDES -------------------------------

\newcommand\var[2]{\newcommand{#1}{\ensuremath{#2}}}
% C'est pratique de définir des variables au début pour éviter de
% réécrire des formules et cela facilite modifier l'affichage.

\makeatletter
\newcommand{\makecustomtitle}{
  \newpage
  \null
  \begin{center}
  \let \footnote \thanks
    {\LARGE \textbf{\@title} \par}
    \vskip 1em
    {\large \@date \par}
    \rule{1.5in}{0.4pt}
  \end{center}
  \par
}
\makeatother


\title{Power supply}
\author{pascal.rainville }
\date{December 2018}

\begin{document}

\maketitle

\section{TODO}

\todo{TEST}

\begin{itemize}
    \item Une boite de métal
    \item Tester de souder des nuts sur ma tôle avec solid-core
    \item Imprimer en 3d un serre fil
    \item Imprimer en 3d des washers?
    \item Pour le vivant et le neutre, mettre des connecteurs avec une gaine protectrice dessus.
    \item Changer le fil électrique pour un plus court.
\end{itemize}

\section{Introduction}

\section{Multimètre}

Connecter mon multimètre dessus avec 3 fils bananes. Ça va me servir d'écran.

Tourner un knob pour choisir ce que le multimètre va afficher. Voltage, Ampérage.

Pas sûr pour l'ampérage c'est dangereux...

\section{Boite de métal}

Je vais avoir une première version qui ne possédera pas tout. Alors, il est nécessaire de pouvoir facilement modifier le panneau avant pour rajouter des éléments.

Idéalement, le panneau du dessous possède des petits rebords avec des nuts pour pouvoir visser.

25cm x 12cm

Comment faire pour que les vis ne dépasse pas?

\section{Serre fil}

Je vais percer le trou dans la tôle du serre fil avec mon forêt à étage.

Diamètre du câble: 9.10mm
donc
Diamètre intérieur: 10mm

Il est composé de 3 pièces. Une pièces principale avec une flange qui reste à l'extérieur et deux méplats pour les deux autres pièces. Les deux autres pièces sont vissés ensemble pour serrer le câble.

Je mettrai du câble électrique pour les trucs trop lousse.

\section{Buck converter}

Mon buck converter prend en entrée 7-32VDC et en sortie 0.8-28VDC. C'est un step down converter. 12A 300W.

\end{document}
