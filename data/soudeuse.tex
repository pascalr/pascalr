\documentclass{article}
\usepackage[utf8]{inputenc}

\title{Soudeuse}
\author{pascal.rainville }
\date{January 2019}

\begin{document}

\maketitle

\section{Contenu}

Je pense acheter un MIG, mais je ne suis pas sûr. Je dois vérifier l'ampérage nécessaire. J'aimerais pouvoir souder 1/16 je crois avec mon MIG.

J'aimerais pouvoir souder de l'aluminium, de l'acier et du stainless. Il faudrait que je vérifie les gaz nécessaires pour chacun d'eux.

Je voudrais savoir le prix aussi.

J'aimerais pouvoir souder MIG et flux core avec ma machine, vérifier si j'ai les galets.

https://www.millerwelds.com/resources/weld-setting-calculators/mig-solid-core-welding-calculator

80 amp est suffisant pour du gauge 18 pour l'acier.
Pour l'inox, ça peut souder du gauge 16 même.

Il faut que je trouve une place pour acheter du métal aussi.

Pour gauge 18:

CO2 18-19 Volts 75\% Argon/25\% CO2: 16-17 Volts

CO2 gas is economical and has deeper penetration on steel, but may be too hot for thin metal. 75\% Argon / 25\% CO2 is better on thin steels, has less spatter and better bead appearance.

Je ne souderai pas de l'aluminium... Ça demande de l'argon comme gaz, et un spool gun idéalement... fuk l'alu

Finalement je pense acheter le tig sur Amazon. 200 amp pour 300\$!!!

%https://www.amazon.ca/Welders-Inverter-Wedling-Equipment-HITBOX/dp/B01HPUX94Q/ref=sr_1_3?ie=UTF8&qid=1546654651&sr=8-3&keywords=tig+welder

Non je ne prendrai pas le TIG sur amazon. Il n'est pas AC/DC, je ne peux pas souder d'aluminium avec. Et je ne peux pas mettre de pédales avec. Un TIG je voudrais en avoir un bon. Un mig pas bon ça ça ne me dérange pas vraiment.
\end{document}
